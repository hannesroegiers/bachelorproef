%%=============================================================================
%% Samenvatting
%%=============================================================================

% TODO: De "abstract" of samenvatting is een kernachtige (~ 1 blz. voor een
% thesis) synthese van het document.
%
% Een goede abstract biedt een kernachtig antwoord op volgende vragen:
%
% 1. Waarover gaat de bachelorproef?
% 2. Waarom heb je er over geschreven?
% 3. Hoe heb je het onderzoek uitgevoerd?
% 4. Wat waren de resultaten? Wat blijkt uit je onderzoek?
% 5. Wat betekenen je resultaten? Wat is de relevantie voor het werkveld?
%
% Daarom bestaat een abstract uit volgende componenten:
%
% - inleiding + kaderen thema
% - probleemstelling
% - (centrale) onderzoeksvraag
% - onderzoeksdoelstelling
% - methodologie
% - resultaten (beperk tot de belangrijkste, relevant voor de onderzoeksvraag)
% - conclusies, aanbevelingen, beperkingen
%
% LET OP! Een samenvatting is GEEN voorwoord!

%%---------- Nederlandse samenvatting -----------------------------------------
%
% TODO: Als je je bachelorproef in het Engels schrijft, moet je eerst een
% Nederlandse samenvatting invoegen. Haal daarvoor onderstaande code uit
% commentaar.
% Wie zijn bachelorproef in het Nederlands schrijft, kan dit negeren, de inhoud
% wordt niet in het document ingevoegd.

\IfLanguageName{english}{%
\selectlanguage{dutch}
\chapter*{Samenvatting}
\selectlanguage{english}
}{}

%%---------- Samenvatting -----------------------------------------------------
% De samenvatting in de hoofdtaal van het document

\chapter*{\IfLanguageName{dutch}{Samenvatting}{Abstract}}

In een tijdperk waarin data een essentiële bron is voor bedrijven, wetenschappelijk onderzoek, en technologische innovatie, is web scraping een belangrijke techniek geworden voor het geautomatiseerd verzamelen van gegevens van websites. Deze thesis onderzoekt een alternatieve methode van webscraping door middel van netwerkverkeersanalyse, een techniek die gegevenspakketten onderschept en analyseert die tussen een browser en een webserver worden uitgewisseld.
\\ \\
De focus van het onderzoek ligt op het ontwikkelen van een Python-script dat in staat is om gestructureerde data uit het netwerkverkeer van websites te extraheren. Dit onderzoek vergelijkt netwerkverkeersanalyse met traditionele web scraping methoden, die vaak beperkt zijn tot het parsen van HTML-code. Traditionele methoden kunnen inefficiënt zijn voor websites die hun inhoud op een dynamische manier laden door de data op te halen via HTTP-verzoeken.
\\ \\
Door gebruik te maken van een proof of concept, gebaseerd op netwerkverkeersanalyse, is aangetoond dat deze methode efficiënter en flexibeler is voor het verzamelen van verborgen of dynamisch geladen data. De proof of concept demonstreert hoe netwerkverkeersanalyse kan worden ingezet om JSON-bestanden te onderscheppen die door websites worden gebruikt om hun inhoud dynamisch te genereren. Dit levert gedetailleerdere en vollediger data op dan traditionele methoden, terwijl het minder verwerkingscapaciteit vergt.
\\ \\
Naast de voordelen van netwerkverkeersanalyse werden ook enkele uitdagingen onderzocht, zoals de invloed van netwerkfluctuaties op de uitvoeringstijd en de complexiteit van het interpreteren van versleutelde of geobfusceerde netwerkverzoeken.
\\ \\
De resultaten van dit onderzoek dragen bij aan de ontwikkeling van nieuwe technieken voor data-extractie en bieden een solide basis voor toekomstig onderzoek naar efficiënte en ethische manieren om webdata te verzamelen.



