%%=============================================================================
%% Conclusie
%%=============================================================================

\chapter{Conclusie}%
\label{ch:conclusie}

% TODO: Trek een duidelijke conclusie, in de vorm van een antwoord op de
% onderzoeksvra(a)g(en). Wat was jouw bijdrage aan het onderzoeksdomein en
% hoe biedt dit meerwaarde aan het vakgebied/doelgroep?
% Reflecteer kritisch over het resultaat. In Engelse teksten wordt deze sectie
% ``Discussion'' genoemd. Had je deze uitkomst verwacht? Zijn er zaken die nog
% niet duidelijk zijn?
% Heeft het onderzoek geleid tot nieuwe vragen die uitnodigen tot verder
%onderzoek?

%Dit onderzoek tracht een antwoord te bieden op de onderzoeksvraag:
%\begin{itemize}
%    \item ''Hoe kan webscraping worden uitgevoerd door middel van netwerkverkeersanalyse?''
%\end{itemize}

Dit onderzoek naar webscraping door middel van netwerkverkeersanalyse heeft aangetoond dat deze techniek een waardevolle aanvulling is op de traditionele methoden voor het extraheren van data van websites. De Proof of Concept, gebaseerd op een Python-script, heeft effectief gestructureerde data uit het netwerkverkeer weten te halen. Dit biedt een alternatieve benadering voor situaties waarin de data niet toegankelijk is via standaard HTML-parsing of wanneer gezocht wordt naar de originele data waar de website gebruik van maakt.
\\ \\
De belangrijkste bevindingen uit dit onderzoek kunnen als volgt worden samengevat:
\begin{itemize}
    \item \textbf{Effectiviteit van Netwerkverkeersanalyse: }Het gebruik van netwerkverkeersanalyse heeft bewezen dat het een efficiënte manier is om data te verzamelen die anders moeilijk bereikbaar is via traditionele web scraping methoden. Dit is vooral relevant voor websites die hun inhoud op een dynamische manier laden door de data op te halen met HTTP-verzoeken.

    \item \textbf{Vergelijking met traditionele web scraping: }Vergeleken met traditionele web scraping biedt netwerkverkeersanalyse een aantal voordelen, zoals een verbeterde efficiëntie en nauwkeurigheid. Door het netwerkverkeer te analyseren, kan relevante data worden geïdentificeerd en verzameld zonder de noodzaak om grote hoeveelheden HTML te parsen, wat tijd en verwerkingskracht bespaart .

    \item \textbf{Uitvoeringstijd: }Het onderzoek heeft ook de uitvoeringstijd van de scraper onderzocht. Secties van de web scraping procedure die afhankelijk zijn van netwerkverkeer vertoonden grotere variaties in uitvoeringstijd. Hier kan uit geconcludeerd worden dat de uitvoeringstijd van de web scraper erg afhankelijk is van de netwerksnelheid en antwoordsnelheid van de server.
\end{itemize}

Samenvattend biedt webscraping via netwerkverkeersanalyse een robuuste en flexibele methode voor het extraheren van data, vooral in complexe scenario's waar traditionele methoden tekortschieten. Verdere ontwikkeling van deze technieken kan leiden tot nog geavanceerdere tools die een breder scala aan dataverzamelingsbehoeften kunnen ondersteunen.
