%==============================================================================
% Sjabloon onderzoeksvoorstel bachproef
%==============================================================================
% Gebaseerd op document class `hogent-article'
% zie <https://github.com/HoGentTIN/latex-hogent-article>

% Voor een voorstel in het Engels: voeg de documentclass-optie [english] toe.
% Let op: kan enkel na toestemming van de bachelorproefcoördinator!
\documentclass{hogent-article}

% Invoegen bibliografiebestand
\addbibresource{voorstel.bib}

% Informatie over de opleiding, het vak en soort opdracht
\studyprogramme{Professionele bachelor toegepaste informatica}
\course{Bachelorproef}
\assignmenttype{Onderzoeksvoorstel}
% Voor een voorstel in het Engels, haal de volgende 3 regels uit commentaar
% \studyprogramme{Bachelor of applied information technology}
% \course{Bachelor thesis}
% \assignmenttype{Research proposal}

\academicyear{2022-2023} % TODO: pas het academiejaar aan

% TODO: Werktitel
\title{Vul hier de voorgestelde titel van je onderzoek in}

% TODO: Studentnaam en emailadres invullen
\author{Hannes Roegiers}
\email{hannes.roegiers@student.hogent.be}

% TODO: Medestudent
% Gaat het om een bachelorproef in samenwerking met een student in een andere
% opleiding? Geef dan de naam en emailadres hier
% \author{Yasmine Alaoui (naam opleiding)}
% \email{yasmine.alaoui@student.hogent.be}

% TODO: Geef de co-promotor op
\supervisor[Co-promotor]{S. Beekman (Synalco, \href{mailto:sigrid.beekman@synalco.be}{sigrid.beekman@synalco.be})}

% Binnen welke specialisatierichting uit 3TI situeert dit onderzoek zich?
% Kies uit deze lijst:
%
% - Mobile \& Enterprise development
% - AI \& Data Engineering
% - Functional \& Business Analysis
% - System \& Network Administrator
% - Mainframe Expert
% - Als het onderzoek niet past binnen een van deze domeinen specifieer je deze
%   zelf
%
\specialisation{Mobile \& Enterprise development}
\keywords{Scheme, World Wide Web, $\lambda$-calculus}

\begin{document}

\begin{abstract}
  Hier schrijf je de samenvatting van je voorstel, als een doorlopende tekst van één paragraaf. Let op: dit is geen inleiding, maar een samenvattende tekst van heel je voorstel met inleiding (voorstelling, kaderen thema), probleemstelling en centrale onderzoeksvraag, onderzoeksdoelstelling (wat zie je als het concrete resultaat van je bachelorproef?), voorgestelde methodologie, verwachte resultaten en meerwaarde van dit onderzoek (wat heeft de doelgroep aan het resultaat?).
\end{abstract}

\tableofcontents

% De hoofdtekst van het voorstel zit in een apart bestand, zodat het makkelijk
% kan opgenomen worden in de bijlagen van de bachelorproef zelf.
%---------- Inleiding ---------------------------------------------------------

\section{Introductie}%
\label{sec:introductie}

In een tijd waarin data als het digitale goud beschouwd wordt, groeit de relevantie van webscraping exponentieel.
Webscraping, een techniek gericht op het geautomatiseerd onttrekken van data van websites, komt voort uit de toenemende 
erkenning van data als drijvende kracht achter zakelijke strategieën, wetenschappelijk onderzoek en artificiële intelligentie.
\\
Bedrijven en organisaties zijn zich steeds meer bewust van de waarde van accurate, actuele data om competitief te blijven
en te anticiperen op veranderende marktomstandigheden. In deze tijd van datagestuurde besluitvorming werkt webscraping als
een krachtige tool om relevevante, real-time data te verkrijgen.
\\
Naast de stijgende behoefte aan gegevens voor zakelijke toepassingen, ondersteunt webscraping ook technologische innovatie. 
De vooruitgang in technologieën zoals artificiële intelligentie, machine learning en big data-analyse vergroot de mogelijkheden
van wat er met de verzamelde gegevens kan worden gedaan.
\\
Binnen deze context is het begrijpen van welke programmeertaal het meest geschikt is voor welke soorten webscraping taken
essentieel. Het doel van deze bachelorproef is een bijdrage leveren aan discussie door Python, Javascript en Java te vergelijken.
Hierdoor kunnen data-engineers en developers een weloverwogen keuze maken bij het implementeren van webscraping projecten
en op die manier optimaal profiteren van de groeiende waarde van data.

%---------- Stand van zaken ---------------------------------------------------

\section{Literatuurstudie}
\label{sec:Literatuurstudie}

De literatuurstudie verkent het domein van webscraping, met specifieke aandacht voor de programmeertalen Python, Javascript en Java,
evenals de verschillende libraries voor deze talen die essentieel zijn voor webscraping scripts.
Deze studie onderzoekt de huidige stand van zaken, richt zich op diverse libraries en identificeert open vragen. 
Deze verkenning vormt de kern van de aankomende vergelijkende studie van programmeertalen en libraries in deze bachelorproef.

\subsection{Webscraping}

\textit{Webscraping}, verwijst naar het automatisch extraheren van informatie en gegevens van websites en deze dan op te slaan in 
in een database of spreadsheet.
Er bestaan een hele hoop webscraping technieken waaronder manuele copy-and paste, matchen van reguliere expressies, HTTP programming
,HTML parsing, DOM parsing, vertical aggregation, webscraping software, semantic notation en computer vision web-page analyserse
\autocite{DeSSirisuriya2015}.

\subsubsection{HTML Parsing}
HTML Parsing houdt in dat de structuur van de HTML documenten wordt geanalyseerd om er specifieke gegevens uit te halen.
Deze techniek werkt het best bij een goed gestructureerde webpagina en maakt gebruik van tags en attributen om de gewenste informatie
te vinden.

\subsubsection{Reguliere exppressies}
Reguliere expressies of regex zijn patronen die gebruikt worden om tekst te matchen op basis van specifieke criteria.
Deze zijn handig voor het zoeken en vinden van informatie met een consistente structuur, zoals telefoonnummers en e-mailadressen.

\subsubsection{API-based webscraping}
Sommige websites bieden Application Programming Interfaces (API's) waarmee gegevens gestructureerd kunnen worden opgevraagd. 
Deze kunnen dus ook gebruikt worden om te webscrapen.

\subsubsection{Selenium}
Selenium is een automatiseringstool waarmee webscraping van dynamische webpagina's mogelijk is, 
waar de inhoud wordt gegenereerd door JavaScript. Selenium simuleert menselijke interactie met een webpagina en is geschikt voor het ophalen van gegevens die via JavaScript worden geladen.

\subsubsection{Machine Learning-based webscraping}
Geavanceerde technieken, zoals machine learning, kunnen worden toegepast om webscraping-modellen te trainen voor het herkennen 
van patronen en het extraheren van specifieke informatie. Het grote voordeel dat webscrapen met Machine Learning met zich meebrengt
is dat het flexibel is en zich makkelijker kan aanpassen aan veranderende webpaginastructuren.

\subsubsection*{}
De keuze van de juiste techniek hangt af van een aantal factoren, waaronder de complexiteit van de webpagina's 
die moeten worden gescraped, de hoeveelheid gegevens die moeten worden geëxtraheerd en de beschikbare middelen.

\subsection{Programmeertalen}
De vergelijking van programmeertalen voor webscraping vormt de kern van deze bachelorproef. 
Door de specifieke kenmerken en mogelijkheden van Python, JavaScript en Java te onderzoeken, 
streven we ernaar inzicht te verkrijgen in welke taal het meest geschikt is voor verschillende webscraping-taken. 
Elk van deze programmeertalen heeft zijn eigen ecosysteem van libraries, frameworks en tools die kunnen worden toegepast 
bij het ontwikkelen van webscrapers.

\subsubsection{Python}
Python staat bekend als een van de meest populaire programmeertalen voor webscraping. Het heeft een uitgebreide gemeenschap en
een overvloed aan beschikbare libraries zoals Beautiful Soup en Scrapy~\autocite{Saabith2019}. De eenvoudige en leesbare syntax van Python maakt het 
gemakkelijk voor zowel beginners als ervaren ontwikkelaars om webscraping-scripts te schrijven.

\subsubsection{}

% Voor literatuurverwijzingen zijn er twee belangrijke commando's:
% \autocite{KEY} => (Auteur, jaartal) Gebruik dit als de naam van de auteur
%   geen onderdeel is van de zin.
% \textcite{KEY} => Auteur (jaartal)  Gebruik dit als de auteursnaam wel een
%   functie heeft in de zin (bv. ``Uit onderzoek door Doll & Hill (1954) bleek
%   ...'')


%---------- Methodologie ------------------------------------------------------
\section{Methodologie}%
\label{sec:methodologie}

Hier beschrijf je hoe je van plan bent het onderzoek te voeren. Welke onderzoekstechniek ga je toepassen om elk van je onderzoeksvragen te beantwoorden? Gebruik je hiervoor literatuurstudie, interviews met belanghebbenden (bv.~voor requirements-analyse), experimenten, simulaties, vergelijkende studie, risico-analyse, PoC, \ldots?

Valt je onderwerp onder één van de typische soorten bachelorproeven die besproken zijn in de lessen Research Methods (bv.\ vergelijkende studie of risico-analyse)? Zorg er dan ook voor dat we duidelijk de verschillende stappen terug vinden die we verwachten in dit soort onderzoek!

Vermijd onderzoekstechnieken die geen objectieve, meetbare resultaten kunnen opleveren. Enquêtes, bijvoorbeeld, zijn voor een bachelorproef informatica meestal \textbf{niet geschikt}. De antwoorden zijn eerder meningen dan feiten en in de praktijk blijkt het ook bijzonder moeilijk om voldoende respondenten te vinden. Studenten die een enquête willen voeren, hebben meestal ook geen goede definitie van de populatie, waardoor ook niet kan aangetoond worden dat eventuele resultaten representatief zijn.

Uit dit onderdeel moet duidelijk naar voor komen dat je bachelorproef ook technisch voldoen\-de diepgang zal bevatten. Het zou niet kloppen als een bachelorproef informatica ook door bv.\ een student marketing zou kunnen uitgevoerd worden.

Je beschrijft ook al welke tools (hardware, software, diensten, \ldots) je denkt hiervoor te gebruiken of te ontwikkelen.

Probeer ook een tijdschatting te maken. Hoe lang zal je met elke fase van je onderzoek bezig zijn en wat zijn de concrete \emph{deliverables} in elke fase?

%---------- Verwachte resultaten ----------------------------------------------
\section{Verwacht resultaat, conclusie}%
\label{sec:verwachte_resultaten}

Hier beschrijf je welke resultaten je verwacht. Als je metingen en simulaties uitvoert, kan je hier al mock-ups maken van de grafieken samen met de verwachte conclusies. Benoem zeker al je assen en de onderdelen van de grafiek die je gaat gebruiken. Dit zorgt ervoor dat je concreet weet welk soort data je moet verzamelen en hoe je die moet meten.

Wat heeft de doelgroep van je onderzoek aan het resultaat? Op welke manier zorgt jouw bachelorproef voor een meerwaarde?

Hier beschrijf je wat je verwacht uit je onderzoek, met de motivatie waarom. Het is \textbf{niet} erg indien uit je onderzoek andere resultaten en conclusies vloeien dan dat je hier beschrijft: het is dan juist interessant om te onderzoeken waarom jouw hypothesen niet overeenkomen met de resultaten.



\printbibliography[heading=bibintoc]

\end{document}