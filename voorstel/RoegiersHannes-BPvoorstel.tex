%==============================================================================
% Sjabloon onderzoeksvoorstel bachproef
%==============================================================================
% Gebaseerd op document class `hogent-article'
% zie <https://github.com/HoGentTIN/latex-hogent-article>

% Voor een voorstel in het Engels: voeg de documentclass-optie [english] toe.
% Let op: kan enkel na toestemming van de bachelorproefcoördinator!
\documentclass{hogent-article}
\usepackage{pgfplots}

% Invoegen bibliografiebestand
\addbibresource{voorstel.bib}

\studyprogramme{Professionele bachelor toegepaste informatica}
\course{Bachelorproef}
\assignmenttype{Onderzoeksvoorstel}

\academicyear{2023-2024} % TODO: pas het academiejaar aan

% TODO: Werktitel
\title{Web Scraping via Netwerkverkeersanalyse: Een alternatieve methode voor data-extractie}

\author{Hannes Roegiers}
\email{hannes.roegiers@student.hogent.be}


% TODO: Geef de co-promotor op
\supervisor[Co-promotor]{}

\specialisation{AI \& Data Engineering}
\keywords{Webscraping, Netwerkverkeersanalyse, JSON, data-extractie, efficiëntie, flexibiliteit, schaalbaarheid, ethiek}

\begin{document}

\begin{abstract}
  Dit onderzoeksvoorstel richt zich op het verkennen van de mogelijkheden voor het webscrapen van netwerkverkeer op websites en bijkomende ethische aspecten. Meer specifiek wordt de focus gelegd
  op het extraheren van JSON-objecten uit het netwerkverkeer aangezien deze de data bevatten. De traditionele webscraping methode laadt de webpagina's in de browser en parst de HTML-code om zo
  te trachten de gewenste data te vinden en op te slaan. Bij webscrapen van het netwerkverkeer wordt vooral het inkomend netwerkverkeer geanalyseerd, maar in sommige gevallen ook het uitgaand verkeer.
  Wanneer een JSON bestand gevonden wordt, wordt deze volledig opgeslaan om dan in een latere fase van de data-extractie er de gewenste data uit te halen. Dit kan gedaan worden aan de hand van proxyservers, API, 
  python libraries of handmatig met de browserontwikkelaarstools. Wanneer men bezig is met webscraping is het belangerijk om voldoende aandacht te besteden aan het ethisch tewerk gaan. Dit onderzoek belicht ook de ethische aspecten
  van webscraping via netwerkverkeersanalyse, met aandacht voor privacy, gegevensbescherming en intellectueel eigendom.
  Het uitvoeren van dit onderzoek vereist ook een Proof of Concept (PoC) om de haalbaarheid en effectiviteit van deze methodes te demonstreren en te evalueren.
  De eerste fase van dit onderzoek bestaat uit een uitgebreide literatuurstudie waarin onderzocht wordt welke webscraping tools ondersteuning bieden voor het analyseren en scrapen van het netwerkverkeer van websites.
  In de tweede fase wordt er gefocusd op het opzetten van een Proof of Concept (PoC) voor de geselecteerde tools. Dit onderzoek beoogt een diepgaand begrip te ontwikkelen van de bruikbaarheid van de
  beschikbare tools en netwerkverkeersanalyse methodes voor webscraping en concrete ethische richtlijnen te formuleren voor het toepassen van deze methoden. Deze paper is relevant voor onderzoekers en professionals die zich interesseren voor webscraping, data-extractie, netwerkverkeersanalyse

\end{abstract}

\tableofcontents

% De hoofdtekst van het voorstel zit in een apart bestand, zodat het makkelijk
% kan opgenomen worden in de bijlagen van de bachelorproef zelf.
%---------- Inleiding ---------------------------------------------------------

\section{Introductie}%
\label{sec:introductie}

In een tijd waarin data als het digitale goud beschouwd wordt, groeit de relevantie van webscraping exponentieel.
Webscraping, een techniek gericht op het geautomatiseerd onttrekken van data van websites, komt voort uit de toenemende 
erkenning van data als drijvende kracht achter zakelijke strategieën, wetenschappelijk onderzoek en artificiële intelligentie.
\\
Bedrijven en organisaties zijn zich steeds meer bewust van de waarde van accurate, actuele data om competitief te blijven
en te anticiperen op veranderende marktomstandigheden. In deze tijd van datagestuurde besluitvorming werkt webscraping als
een krachtige tool om relevevante, real-time data te verkrijgen.
\\
Naast de stijgende behoefte aan gegevens voor zakelijke toepassingen, ondersteunt webscraping ook technologische innovatie. 
De vooruitgang in technologieën zoals artificiële intelligentie, machine learning en big data-analyse vergroot de mogelijkheden
van wat er met de verzamelde gegevens kan worden gedaan.
\\
Binnen deze context is het begrijpen van welke programmeertaal het meest geschikt is voor welke soorten webscraping taken
essentieel. Het doel van deze bachelorproef is een bijdrage leveren aan discussie door Python, Javascript en Java te vergelijken.
Hierdoor kunnen data-engineers en developers een weloverwogen keuze maken bij het implementeren van webscraping projecten
en op die manier optimaal profiteren van de groeiende waarde van data.

%---------- Stand van zaken ---------------------------------------------------

\section{State-of-the-art}
\label{sec:state-of-the-art}

De literatuurstudie verkent het domein van webscraping, met specifieke aandacht voor de programmeertalen Python, Javascript en Java,
evenals de verschillende libraries voor deze talen die essentieel zijn voor webscraping scripts.
Deze studie onderzoekt de huidige stand van zaken, richt zich op diverse libraries en identificeert open vragen. 
Deze verkenning vormt de kern van de aankomende vergelijkende studie van programmeertalen en libraries in deze bachelorproef.

\subsection{Webscraping}

\textit{Webscraping}, verwijst naar het automatisch extraheren van informatie en gegevens van websites en deze dan op te slaan in 
in een database of spreadsheet.
Er bestaan een hele hoop webscraping technieken waaronder manuele copy-and paste, matchen van reguliere expressies, HTTP programming
,HTML parsing, DOM parsing, vertical aggregation, webscraping software, semantic notation en computer vision web-page analyserse
\autocite{DeSSirisuriya2015}.

% Voor literatuurverwijzingen zijn er twee belangrijke commando's:
% \autocite{KEY} => (Auteur, jaartal) Gebruik dit als de naam van de auteur
%   geen onderdeel is van de zin.
% \textcite{KEY} => Auteur (jaartal)  Gebruik dit als de auteursnaam wel een
%   functie heeft in de zin (bv. ``Uit onderzoek door Doll & Hill (1954) bleek
%   ...'')


%---------- Methodologie ------------------------------------------------------
\section{Methodologie}%
\label{sec:methodologie}

Hier beschrijf je hoe je van plan bent het onderzoek te voeren. Welke onderzoekstechniek ga je toepassen om elk van je onderzoeksvragen te beantwoorden? Gebruik je hiervoor literatuurstudie, interviews met belanghebbenden (bv.~voor requirements-analyse), experimenten, simulaties, vergelijkende studie, risico-analyse, PoC, \ldots?

Valt je onderwerp onder één van de typische soorten bachelorproeven die besproken zijn in de lessen Research Methods (bv.\ vergelijkende studie of risico-analyse)? Zorg er dan ook voor dat we duidelijk de verschillende stappen terug vinden die we verwachten in dit soort onderzoek!

Vermijd onderzoekstechnieken die geen objectieve, meetbare resultaten kunnen opleveren. Enquêtes, bijvoorbeeld, zijn voor een bachelorproef informatica meestal \textbf{niet geschikt}. De antwoorden zijn eerder meningen dan feiten en in de praktijk blijkt het ook bijzonder moeilijk om voldoende respondenten te vinden. Studenten die een enquête willen voeren, hebben meestal ook geen goede definitie van de populatie, waardoor ook niet kan aangetoond worden dat eventuele resultaten representatief zijn.

Uit dit onderdeel moet duidelijk naar voor komen dat je bachelorproef ook technisch voldoen\-de diepgang zal bevatten. Het zou niet kloppen als een bachelorproef informatica ook door bv.\ een student marketing zou kunnen uitgevoerd worden.

Je beschrijft ook al welke tools (hardware, software, diensten, \ldots) je denkt hiervoor te gebruiken of te ontwikkelen.

Probeer ook een tijdschatting te maken. Hoe lang zal je met elke fase van je onderzoek bezig zijn en wat zijn de concrete \emph{deliverables} in elke fase?

%---------- Verwachte resultaten ----------------------------------------------
\section{Verwacht resultaat, conclusie}%
\label{sec:verwachte_resultaten}

Hier beschrijf je welke resultaten je verwacht. Als je metingen en simulaties uitvoert, kan je hier al mock-ups maken van de grafieken samen met de verwachte conclusies. Benoem zeker al je assen en de onderdelen van de grafiek die je gaat gebruiken. Dit zorgt ervoor dat je concreet weet welk soort data je moet verzamelen en hoe je die moet meten.

Wat heeft de doelgroep van je onderzoek aan het resultaat? Op welke manier zorgt jouw bachelorproef voor een meerwaarde?

Hier beschrijf je wat je verwacht uit je onderzoek, met de motivatie waarom. Het is \textbf{niet} erg indien uit je onderzoek andere resultaten en conclusies vloeien dan dat je hier beschrijft: het is dan juist interessant om te onderzoeken waarom jouw hypothesen niet overeenkomen met de resultaten.



\printbibliography[heading=bibintoc]

\end{document}